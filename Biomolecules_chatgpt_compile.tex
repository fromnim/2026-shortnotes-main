\documentclass[12pt]{article}
\usepackage[utf8]{inputenc}
\usepackage{amsmath, array, booktabs, mhchem}
\renewcommand{\arraystretch}{1.2}

\title{Biomolecules NCERT Notes}
\date{}

\begin{document}
\maketitle

Carbohydrates are polyhydroxy aldehydes or ketones, or compounds that yield them on hydrolysis. Also called saccharides.

\section*{Preparation of Glucose}
\begin{itemize}
  \item \textbf{From sucrose:} Boil with dilute \ce{H2SO4} or \ce{HCl}. Forms glucose + fructose.
  \item \textbf{From starch:} Hydrolysis with dilute \ce{H2SO4} at 300 K gives glucose.
\end{itemize}

\section*{Structure of Glucose}
\begin{itemize}
  \item Forms n-hexane with \ce{HI} $\Rightarrow$ 6 carbons.
  \item Forms oxime, cyanohydrin $\Rightarrow$ carbonyl group.
  \item With Br\textsubscript{2} water $\Rightarrow$ gluconic acid (has \ce{-CHO}).
  \item Acetylation $\Rightarrow$ pentaacetate $\Rightarrow$ 5 \ce{-OH} groups.
  \item With \ce{HNO3} $\Rightarrow$ saccharic acid $\Rightarrow$ \ce{-CH2OH} present.
\end{itemize}

\section*{Cyclic Glucose}
\begin{itemize}
  \item No Schiff's test.
  \item Pentaacetate doesn't react with hydroxylamine $\Rightarrow$ no free \ce{-CHO}.
  \item Has $\alpha$ and $\beta$ forms (anomers).
  \item 6-membered ring $\Rightarrow$ pyranose.
\end{itemize}

\section*{Key Points}
\begin{itemize}
  \item Sucrose $\rightarrow$ D-(+)-glucose + D-(-)-fructose. Net: laevorotatory.
  \item Starch = amylose + amylopectin.
  \item Cellulose and glycogen = polysaccharides.
\end{itemize}

\section*{Amino Acids and Proteins}
\begin{itemize}
  \item Acidic/basic: based on \ce{-NH2} and \ce{-COOH} count.
  \item D/L-form: \ce{NH2} right = D, left = L.
  \item Proteins = polypeptides with peptide bonds (\ce{CO-NH}).
  \item >10 amino acids = polypeptide.
\end{itemize}

\subsection*{By Shape}
\begin{itemize}
  \item \textbf{Fibrous:} Long, structural (e.g., keratin).
  \item \textbf{Globular:} Spherical, functional (e.g., enzymes).
\end{itemize}

\subsection*{Protein Structure}
\begin{itemize}
  \item Primary: Amino acid sequence.
  \item Secondary: $\alpha$-helix or $\beta$-sheet.
  \item Tertiary: 3D folding.
  \item Quaternary: Multiple chains.
\end{itemize}

\subsection*{Properties}
\begin{itemize}
  \item Denaturation = loss of function (e.g., heat).
  \item Amino acids = zwitterions: high solubility, high mp.
\end{itemize}

\section*{Enzymes}
\begin{itemize}
  \item Globular proteins acting as biocatalysts.
  \item Specific (e.g., maltase on maltose).
  \item Oxidoreductases = redox catalysts.
\end{itemize}

\section*{Vitamins}
\begin{itemize}
  \item \textbf{Fat-soluble:} A, D, E, K
  \item \textbf{Water-soluble:} B-complex, C
  \item A: Xerophthalmia, B\textsubscript{1}: Beri-beri, B\textsubscript{2}: Cheilosis
\end{itemize}

\section*{Starch}
\begin{itemize}
  \item Amylose: 10--15\%, linear, 1$\rightarrow$4, soluble.
  \item Amylopectin: 85--90\%, branched (1$\rightarrow$4, 1$\rightarrow$6), insoluble.
\end{itemize}

\section*{Sucrose Linkage}
\begin{itemize}
  \item 1$\rightarrow$2 glycosidic bond (glucose--fructose).
  \item Hydrolysis: \ce{H2O/H+} notation.
\end{itemize}

\section*{Fibrous vs Globular Proteins}

\begin{tabular}{@{}p{4cm}p{5cm}p{5cm}@{}}
\toprule
\textbf{Feature} & \textbf{Fibrous} & \textbf{Globular} \\
\midrule
Shape & Long & Spherical \\
Water Solubility & Insoluble & Soluble \\
Function & Structural & Functional \\
Examples & Keratin, collagen & Hemoglobin, enzymes \\
Structure & Repetitive, simple & Complex folding \\
Stability & High & Low \\
\bottomrule
\end{tabular}

\end{document}
