\documentclass[12pt]{article}
\usepackage{mhchem}
\title{AK&CA}
\begin{document}
\maketitle

\section*{Preparation}
By reduction: Simply reduce carboxylic acid.
By dehydrogenation: Alcohol is given copper and heat.
By ozonolysis: Draw the O atoms in a house like shape and 
then slash the molecule into two pieces.
By hydrating alkynes: By adding H3O+/HgSO4 to alkenes one 
can get markonikov addition of water in the alkene.

\section*{Preparation of RCHO and R'CHO only}
\begin{itemize}

\item Rosenmund reaction: Pd + BaSO4 is used to create 
aldehyde 
\item Stephen reaction(They call it stephen's reduction
of cyanide): Use SnCl2+HCl then water on nitrile. This will
lead to markonikov addition of water creating a alkanamide
which will be hydrolysed to give us aldehyde.
\item Dibal-H: Di-isobutyl aluminium hydride. What Stephens 
reagent does in a few steps.. dibalH does in one{two actually 
hydrolysis is needed}..
\item Using amide: Adding hydrogen. create aldehyde with NH3
gas
\item Oxidise alcohol to aldehyde
\item Gatterman koch reaction: Use CO + HCl to add CO to benzene
before watching the H+ add once again to the benzene.
\item Etard reaction: It is capable of converting toluene to 
to aldehyde.. which is great. A brown coloured complex is formed
in between of the product of toulene and chromyl chloride.
In CrO2Cl2 one of the Cl takes hydrogen from the toluene and forms CH4
while the rest of the molecule is attached to the carbon of touene.
\item Hydrolysis: Hydrolyse RCN to give aldehyde.
\item Special reaction: Reimer tiemann reaction: Phenol to saclicylaldehyde
\item Ester is converted to ketone and alcohol.

\end{itemize}
\section{Preparation of Ketones}
\begin{itemize}

\item 