\documentclass[12pt]{article}
\usepackage{mhchem}
\title{Amines from NCERT and random notes i found online}
\begin{document}
\maketitle

Preparation of Amines:
\begin{itemize}

\item Reduction of Nitro compounds: When nitro compounds are reduced
we get amines, they can be both benzylic and alkylic{if that's a word}

Fe + HCl is preffered mode of creating amines in these methods
because it hydrolyses to release HCl during reaction in situ. Therefore
fresh supply of HCl incoming all the time.

\item Ammonolysis: As the name suggests, we break a bond with NH3
molecule of RX and settle ourselves there at 373K. Because we formed
a RNH2 yet we can add more R's to it.. we do that and end up with
${R_{4}N}^{+}$$X^{-}$. Which is then turned into amine with the 
help of NaOH. The reaction for that is: \ce{${R_{4}N}^{+}$$X^{-}$ + NaOH -> $4RNH_{2}$ + H_{2}O + NaX}.
Which is great.
Disadvantage of this is that we get a mix of all degree products
which is difficult to separate. This is also not valid for aryl amine

\item Reducing nitriles: Simply hydrogenation of CtriplebondN.

\item Reducing amines: Amide when reduced loses the carbonyl carbon
entirely and the main carbon now has hydrogens to fill the valency.

\item Gabriel pthalimide synthesis: We take RX and add some of that 
pottasium pthaline ion. This will create N-R group from N-K in the
pthaline ion.. now we add NaOH, which will cleave the bonds between
carbonyl carbons and the nitrogen atom. Finally we are left with a 
RNH2 and pthaline ion which you may predict but having ONa instead of 
H atoms.{If i would put the bond cleavage part differently i wouldsay..
"Two NaOH molecules break to give 2$ONa^{-}$ and 2$H^{+}$ which will
then bond with the pthaline ion and the amine respectively"}
Aromatic primary amines cannot be prepare with Gthis method because
Aryl halides do not undergo nucleophilic substitution.

\item Hoffman bromamide degradation: It is used to prepare amines 
from amide using bromine gas in NaOH.. the end by products are Na2CO3
 and NaBr along with water.

\item Special reaction: alkylammonium salt is reacted with NaOH.
This will give RNH2 + NaX + H2O

\end{itemize}
\section*{Chemical Properties of anime}
\begin{itemize}

\item Friedel crafts alkylation: \ce{RNH_{2} + CH_{3}Cl -> R-NH-CH_{3} + HCl}

\item  Acylation: Primary amine with acid chloride gives amide 
and HCl molecule Similarly when we add secondary amine to acid 
annhydride we get carboxylic acid and amide.

\item Benzoylation: Reaction of amine with benzoyl chloride at room
temperature. 

\item Carbylamine reaction: It is used to test for primary amine
where the amine is reacted with chloroform and base to form isocyanide{foul smell}
It may also be used to distinguish between CCl4 and CHCl3.. as both 
are very similar to each other. The reaction mechanism is that: CHCl3
loses its H to $OH^{-}$ and forms $CCl_{3}^{-}$ which will form 
dichlorocarbene. This dichlorocarbene attacks the amine and creates
the isocyanide with steps i do not know about.

\item reaction with nitrous acid: Aniline can form diazonium salt.
This salt when hydrolysed is converted to phenol and releases N2 
gas

\item Hinsberg reagent: Test for *ary amine. When primary amine
reacts with hinsberg reagent(it is PhSO2Cl) also called benzene
sulphonyl chloride, the sulphur atom breaks bond with Cl to remake 
it with the Nitrogen atom of amine which has now lost a hydrogen.
This new thing is now added to alkali and we find it to be soluble in
it..(because there is a alpha H for making salt)
Secondary amine does react but due to lack of alpha hydrogen atom 
it fails to be soluble.
Tertiary amine does not give reaction.

\item Bromination of Aniline: When aniline is brominated, it over 
brominates and forms a white ppt of 2,4,6-tribromoanilinie{changing 
the colour from reddish brown to white}. To prevent this thing from
happening we will do acetylation of the nitrogen group(essentially
adding a EWG.) And remove the same after our bromine is added by
hydrolysis.

\item Nitration of aniline: Before the nitration even takes place
using the concH2SO4 and concHNO3, we get a anilinium ion on the
ring. This anilinium ion now in its ortho and para products of NO2
is stabler in the para one.. because when in ortho{i dont know the 
reason.}

\item Sulphonation of aniline: When aniline is reacted with concH2SO4
we get a salt with $NH_{3}^{+}SO_{4}^{-}$ which at high temperature 
breaks down to get us para and ortho "4-paranitro sulphonic acid which 
also behaves as a zwitter ion.

\item Why aniline cannot give friedel crafts reaction? Because when it 
tries to do that.. the strong lewis acid AlCl3 ends up reacting with 
the lone pair of nitrogen on aniline instead.

\item Smaller the size more the solvation.

\item Chemical prop of Diazonium: 1. $N_{2}Cl^{-}$ can be replaced with 
Cl or Br by adding HX.2. Replace with CN by adding HCN.3. Replace by OH 
by adding water.4. Replace with H by adding H3PO2.5. Replace with F by 
adding HBF4.6. Replace with I by adding KI.6. Replacement by NO2 can be 
done by adding HBF4 and then NaNO2 so that in the end only NO2 is there.
with by products being the NaBF4 and HCl.

\item Coupling reaction of azo dye test: benzene diazoniumsalt is taken with
 a electron donating group based aromatic compound.. and this compound will
lose its para hydrogen to the Cl- of the diazonium to produce HCl and another
compound as a result of both nitrogen atoms being distributed between the two 
aromatic compounds with a double bond in between.

\item can be replaced by COOH but don't add CH3COOH directly to diazonium 
that wont do a nything.

\item Most conversions start with NO2 which is then converted to NH2 and then
we can easily create diazonium and get other substitutents

\item Do the benzoic acid to aniline conversion

\item \ce{CH_{3}NH_{2} ->[HNO2 and HCl] CH_{3}OH}

\item To reduce carbon atom: Hofmann bromamide degradation(remove a C=O whole)

\end{itemize}
\end{document}























