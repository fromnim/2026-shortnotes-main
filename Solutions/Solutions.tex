\documentclass{article}
\begin{document}

\section*{maximum boiling aezotrope and deviation}
When we have a max boiling aezotrope we know that it has a lower vapour pressure than its constitutents, this is because for it to reach the boiling point after its constituents it has to reach the vapour pressure of boiling before them. Therefore it is proven to be a case of negative deviation.

Some points i learned
\begin{itemize}

\item At equilibrium for a reaction of $\ce{A(l) -> A(g)}$. The Kp of reaction is equal to the vapour pressure. Thus we can say that the vapour pressure is equilibrium constant, which makes its value only dependant on temperature.

\item Just like there is the Van't hoff equation for the variation of $K_{c}$ with temperature. We have the Clausius clayperon equation for the variation of Kp{aka vapour pressure} with temperature.
