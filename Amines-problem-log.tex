\documentclass{article}
\usepackage{mhchem}
\title{Amines-problem-log}
\begin{document}
\maketitle

\section*{Table of reagents}
\begin{tabular}

Hofmann bromamide degradation & removes a C=O whole from amide \\
Benzoic acid to & benzene and CO2 \\
COOH can be converted to COCl using & SOCl2 \\
Benzoyl chloride can be converted to benzonitrile using & NaCN \\
Benzoic acid when heated with ammonia(aq) gives & Benzamide \\
When CHCl3 and 3KOH is added to aniline we get & Isocyanide group instead of NH2 \\
Reagent for carbylamine reaction is the same as.. & Reimer tiemann reaction \\
Carbylamine reaction(also given by aniline) is used as what test? & Test for primary amine \\
When alcohol is added to diazonium salt and heated it converts to? & Benzene \\
when we add H3PO2 to the diazonium salt we get & benzene. \\
When diazonium is treated with KI we get & Iodobenzene \\ 
Partial hydrolysis of CN give & Amide \\
HBF4 can be used on dizonium salt to create NO2 substitusnt & using NaNO2 and Cu in between \\
  

\section*{Points to remember}
\begin{itemize}

\item Ammmonolysis of alkyl halide creates a mix of all degree
products which does not mean anything.

\item LiAlH4 converts nitrobeginzene to pure azo compound instead of
aniline. Which is why we don't use it.

\item Think about the reactants fundamentallly. 

\item hydrazine is used at the end of the gabriel synthesis reaction 
because it is simply better.

\item Solubility is based on how well you can create hydrogen bonds with 
the existing solvent..(atleast in the case of hydrogen bonding)

\item The sulphanilic acid's zwitter ion structure was directly 
asked.

\end{itemize}

\section*{Azo dyes}

\includegraphics{fig1.png}

The above shown reaction is to form phenol oragne dye.

\includegraphics{fig2.png}

This is the formation of aniline yellow

\includegraphics{fig3.png}

This is the formation of beta napthol red.


\item a primary amine is more basic than a tertiary in aq solution because
the primary amine's conjugate base can form hydrogen bonds(as it has a hydrogen atom)
while the tertiary on top of having no hydrgen's bonded also lacks steric stability.

\item carbylamine reaction involved amine being attacked by dicholorocarbene which makes 
nitrogen lose one of its electrons which settles on the newly added carbone of DCC.. this
intermediate loses another hydrogen bonded to nitrogen, chlroine bonded to carbon and forms
a double bond between N and C. Finally the last chlorine atom leaves by itself, giving us the 
isocyanide group.
		
\item The meta product of nitration of aniline is formed in considerable amounts because in any 
other EAS there would be a positive charge very close to the NH3+ .. whcih is very unstable.

\item All the complexity of aniline can be ignored if we just acetalyse it and reduce the reactivity.

